%%%%%%%%%%%%%%%%%%%%%%% file typeinst.tex %%%%%%%%%%%%%%%%%%%%%%%%%
%
% This is the LaTeX source for the instructions to authors using
% the LaTeX document class 'llncs.cls' for contributions to
% the Lecture Notes in Computer Sciences series.
% http://www.springer.com/lncs       Springer Heidelberg 2006/05/04
%
% It may be used as a template for your own input - copy it
% to a new file with a new name and use it as the basis
% for your article.
%
% NB: the document class 'llncs' has its own and detailed documentation, see
% ftp://ftp.springer.de/data/pubftp/pub/tex/latex/llncs/latex2e/llncsdoc.pdf
%
%%%%%%%%%%%%%%%%%%%%%%%%%%%%%%%%%%%%%%%%%%%%%%%%%%%%%%%%%%%%%%%%%%%


\documentclass[runningheads,a4paper]{llncs}

\usepackage{amssymb}
\setcounter{tocdepth}{3}
\usepackage{graphicx}

\usepackage{url}
\urldef{\mailsa}\path|{tpatzelt, dpfuetze, untergasser, sluytergaeth}@uni-potsdam.de|  
\newcommand{\keywords}[1]{\par\addvspace\baselineskip
\noindent\keywordname\enspace\ignorespaces#1}

\begin{document}

\mainmatter  % start of an individual contribution

% first the title is needed
\title{Automatic Text Simplification using a Siamese Convolutional Neural Network}

% a short form should be given in case it is too long for the running head
\titlerunning{Automatic Text Simplification using Siamese CNN}

% the name(s) of the author(s) follow(s) next
%
\author{Tim Patzelt, Dominik Pfuetze, Henny Sluyter-Gaethje, Simon Untergasser%
%\thanks{Potsdam University}
%\thanks{Matr-Nr. 794769, }%
}
%
\authorrunning{Automatic Text Simplification using Siamese CNN}
% (feature abused for this document to repeat the title also on left hand pages)

% the affiliations are given next; don't give your e-mail address
% unless you accept that it will be published
\institute{Potsdam University\\
Matr.-Nr. 794948, 794948\\
\mailsa
}

%
% NB: a more complex sample for affiliations and the mapping to the
% corresponding authors can be found in the file "llncs.dem"
% (search for the string "\mainmatter" where a contribution starts).
% "llncs.dem" accompanies the document class "llncs.cls".
%

\toctitle{Automatic Text Simplification using Siamese CNN}
\maketitle


\begin{abstract}
Text simplification is the act of reducing the linguistic complexity of a text while preserving the information expressed in the text. Many approaches to perform this task
automatically rely on the concept of neural machine translation. In our work we investigate whether having both a complex and a simple sentence as input for a neural network improves over networks having only one sentence as input. 

\keywords{Text Simplification, Siamese Network, CNN}
\end{abstract}


\section{Introduction}
\section{Related work}
\section{Approach}
\subsection{ABCNN}
\subsection{Data and preprocessing}

\subsection{Changes to the network}

\section{Results}

\section{Conclusion}

\section{Future work}

\bibliography{lit}
\bibliographystyle{alpha}
\end{document}